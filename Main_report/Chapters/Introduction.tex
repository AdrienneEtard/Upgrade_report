% Introduction to thesis
Anthropogenic activities are driving global biodiversity declines at unprecedented rates. Currently, habitat conversion and degradation -- induced mainly by anthropogenic land-use change --  are the primary causes of biodiversity loss (Pereira, Navarro and Martins, 2012; Newbold et al., 2015). Climate change is projected to be one of the biggest driver of biodiversity loss by 2070, matching or exceeding the deleterious impacts of land-use change on ecological communities (Newbold, 2018). Understanding how land-use and climate change (LUCC) act on biodiversity, separately and in combination, is key to project the future responses of species, and to consequently put into place efficient policies for biodiversity conservation. Furthermore, biodiversity losses affect ecosystem properties, and can adversely impact the delivery of ecosystem services (REF). Investigating if and how biodiversity decreases link to the loss of ecosystem functions is a key research area (Petchey and Gaston, 2006; Lefcheck et al., 2015) and can help mitigate the impacts of anthropogenic activities on ecosystem processes and services.
It has now been established across diverse taxonomic groups that species traits mediate species responses to environmental changes, notably to LUCC (Newbold et al., 2013; Pearson et al., 2014; Pacifici et al., 2017; Estrada et al., 2018). McGill et al. (2006) defined traits as characteristics of organisms, measurable at the level of an individual across species. This definition can be broadened to include “ecological” traits, where species relation to their surrounding environment needs to be considered. Functional traits are those that particularly influence organismal fitness. Functional traits relate to species’ abilities to exploit their biotic and abiotic environment and as such, shape ecosystem processes. Functional traits underpin both species’ aptitudes to cope with environmental changes and their role in ecosystem functioning (Díaz et al., 2013). Specifically, ‘response traits’ affect species responses to disturbances, while ‘effect traits’ shape ecosystem processes. Certain traits can act as both effect and response traits. Conceptually, these are particularly interesting for investigating the impact of environmental changes on ecosystem processes and services, as they provide a mechanistic understanding of how stressors affect both species’ responses and ecosystem processes (Luck et al., 2012; Hevia et al., 2017).  Assessing the impacts of human activities on ecosystem functioning is increasingly important as pressures rise globally. Publications linking drivers of change and delivery of ecosystem services have increased exponentially since 2001 (Hevia et al., 2017); nevertheless, how species traits influence their responses to land-use and climate change, and how this relates to the loss of important ecosystem functions, remains to be largely explored.
The aim of this PhD project is to investigate the effects of terrestrial vertebrate species’ traits on their responses to LUCC, at global scales. Specifically, my main goals are (1) to elucidate which traits are likely to put species at greater risk from land-use and climate change; (2) to investigate whether future biodiversity declines triggered by these anthropogenic threats are likely to disrupt important ecosystem functions. Unlike previous published studies, this work will investigate these questions at a global scale, and simultaneously across the four terrestrial vertebrate classes – amphibians, birds, reptiles and mammals. 
This report synthetizes the work I have achieved throughout the first year. I start by briefly reviewing the literature to present the questions I have addressed in the context of the past and current ecological research (Chapter 1). Chapter 2 exposes the methods and results of data collection. Chapter 3 investigates how the functional diversity of vertebrate communities is affected by land-use change. Finally, I present an outline of the questions that I plan to investigate in the upcoming years.
