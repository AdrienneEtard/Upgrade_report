% Introduction to thesis
Anthropogenic activities are driving global biodiversity declines at unprecedented rates. Currently, habitat conversion and degradation -- induced mainly by anthropogenic land-use change -- are the primary causes of biodiversity loss \citep{Pereira2012, Newbold2015}. Climate change is projected to be one of the biggest driver of biodiversity loss by 2070, matching or exceeding the deleterious impacts of land-use change on ecological communities \citep{Newbold2018}. Understanding how land-use and climate change (LUCC) act on biodiversity, separately and in combination, is key to project the future responses of species, and to consequently put into place efficient policies for biodiversity conservation. Furthermore, biodiversity losses affect ecosystem properties, and can adversely impact the delivery of ecosystem services \citep{Hooper2012, Oliver2015,MEA2005}. Assessing if and how biodiversity decreases link to the loss of ecosystem functions is a key research area \citep{Petchey2006, Lefcheck2015}. Indeed, investigating how anthropogenic activities are likely to impact biodiversity, and how these impacts may relate to ecosystem processes, can help put into place mitigation measures aiming at protecting both biodiversity and ecosystem functioning.

Although there remains much uncertainty, it has now been established across diverse taxonomic groups that species traits mediate species responses to environmental changes, notably to LUCC \citep{Newbold2013, Pearson2014,Pacifici2017,Estrada2018, Angert2011}.
\citet{McGill2006} defined traits as well-defined organismal characteristics, that can be measured at the individual level and that be used comparatively across species. Functional traits are those that particularly influence organismal fitness or performance \citep{Violle2007}. Functional traits relate to species' abilities to exploit their biotic and abiotic environment. Functional traits can be divided into two groups. Functional traits that underpin species contributions to ecosystem processes have been termed `effect traits' \citep{Lavorel2002, Wong2018}.  Effect traits determine how species use environmental resources and influence ecosystem processes. On the other hand, `response traits' shape how species respond to disturbances. Therefore, functional traits underpin both species' aptitudes to cope with environmental changes and their role in ecosystem functioning.  Certain functional traits can act as both effect and response traits. Conceptually, these are particularly interesting for investigating the impacts of environmental changes on ecosystem processes and services, as they provide a mechanistic understanding of how stressors affect both species' responses and ecosystem processes \citep{Lavorel2002, Luck2012, Hevia2017}.
  
Assessing the impacts of human activities on ecosystem functioning is increasingly important as pressures rise globally. Publications linking drivers of change and delivery of ecosystem services have increased exponentially since 2001 \citep{Hevia2017}; nevertheless, how species traits influence their responses to land-use and climate change, and how this relates to the loss of important ecosystem functions, remains to be largely explored.  Notably, most studies investigating these questions have been conducted at local or regional scales on given taxonomic groups. Although response traits to LUCC have been identified in terrestrial vertebrates, there still remains uncertainty, for example as to which are most important to define species responses \citep{Wheatley2017}. In terrestrial vertebrates, no study has, to my knowledge, attempted to investigate how anthropogenic pressures are likely to disrupt ecosystem processes supported by terrestrial vertebrates at global scales.
 
The final aim of this PhD project is to investigate the effects of terrestrial vertebrate species' traits on their responses to LUCC, at global scales. Specifically, my main goals are (1) to elucidate which traits are likely to put species at greater risk from land-use and climate change; (2) to investigate whether future biodiversity declines triggered by these anthropogenic threats are likely to disrupt important ecosystem functions. Unlike previous published studies, this work will investigate these questions at global scales, and simultaneously across the four terrestrial vertebrate classes -- amphibians, birds, reptiles and mammals.  

This report synthesizes the work I have achieved throughout the first year of my PhD. I start by reviewing the literature to present the questions I have addressed in the context of the past and current ecological research (Chapter 1). In Chapter 2 and 3, I expose the work achieved so far. Specifically, Chapter 2 focuses on the collation of extensive trait data across terrestrial vertebrates. Using these data, Chapter 3 presents a first analysis showing that land-use change promotes the functional homogenisation of local vertebrate communities. Finally, I present an outline of the questions that I plan to investigate in the upcoming years in Chapter 4.

