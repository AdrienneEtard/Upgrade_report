\section{Introduction}
%% Chapter 3: functional diversity analyses
%Dormann 2012 Collinearity: a review of methods to deal with it and a simulation study evaluating their performance
% Mouillot et al 2014 
\section{Methods}
\subsection{Trait selection for functional diversity analyses}
All traits were considered for inclusion into the calculation of functional diversity metrics, except variables relating to species diet, as these were unavailable for reptiles. Deciding which trait to select was a critical step, as functional metrics can be sensitive to the number of traits included (Mouillot et al 2014). Multicollinearity in the included traits can create a form of redundancy in the metrics, hence selecting traits that were not too correlated was important. To this end, I used a stepwise selection procedure to identify potential problematic variables, based on Variance Inflation Factors. Before the stepwise selection procedure, traits were transformed and z-scored (centred to mean value and unit variance). A log-10 transformation was applied to all continuous traits except habitat breadth, which was square-rooted. The stepwise selection procedure consisted in regressing in each continuous traits against each other, then calculating variance inflation scores. With a value 

% Or another approach: facotr analysis of mixed type data

\section{Results}

\section{Discussion}
% sensitivity to trait inclusion
% traits that were not considered that could be important, noably home range, dispersal abilities, or volancy -- traits relating to species abilities to move.