%%% LITERATURE REVIEW

\section{Land-use and climate change, species traits and the functional composition of vertebrate communities}
Currently, terrestrial land-use change is the most important driver of biodiversity declines (Newbold et al., 2015; Chaudhary and Kastner, 2016). With climate change projected to be catching up by 2070 (Newbold, 2018), it has become vital to understand how these threats will affect biodiversity, separately and in combination. By influencing species responses to environmental changes, response traits can provide a mechanistic understanding of how diverse threats shape ecological communities, an understanding particularly relevant for conservation policies. 
There is now empirical evidence across taxonomic groups that species traits influence their responses to LUCC. For instance, response traits to LUCC have been identified in terrestrial plant (Díaz, Noy-Meir and Cabido, 2001), fungal (Koide et al 2013), invertebrate (Williams et al., 2010; Hall et al., 2019), and vertebrate assemblages (Table 1). Overall, these studies, conducted on different taxa and at different scales, tend to show that larger, longer-lived specialist species with a lower reproductive output are more likely to be impacted negatively by LUCC (Table 1). Nevertheless, it is important to point out that in some cases, contrasting results are found (REF); this highlights the fact that studies may be context-dependent, with contingent limitations. 
As response traits determine whether a species is likely to be removed from a community due to the environmental filtering exerted by a pressure, changes in the distribution of trait values are expected in vertebrate communities that have faced LUCC. As such, some studies have documented changes in the trait composition of vertebrate communities that have faced climate change or alongside land-use gradients. Some studies summarise changes in the trait community composition using functional diversity indices (Flynn et al., 2009), which describe the diversity and variation of trait values across organisms. Other studies document changes in the distribution of values for particular traits (Rapacciuolo et al., 2017; La Sorte et al., 2018), which can be seen as a particular use of functional diversity indices when only one trait is considered. 
All these studies show that (1) anthropogenic LUCC is reshaping the functional composition of ecological assemblages, potentially disrupting important functions; (2) species sensitivity to LUCC depends on their traits. Nevertheless, despite this empirical evidence, there is still a need to refine our understanding of which traits significantly influence responses. As traits are commonly used to assess species vulnerability to threats or extinction risks (Pacifici et al., 2015; Willis et al., 2015; Bohm et al., 2016), it is particularly important to be confident about how they act on species responses. The interest for trait-based approaches highlights their potential to inform conservation policies. Trend-based approaches require important field work effort to monitor species populations. Getting extensive information on all species population trends is virtually impossible. The appeal of trait-based approaches is that, by providing mechanistic insights, they diminish the amount of population information needed. If species’ responses to a threat consistently relate to certain traits, it is possible to generalise patterns across species for which data is less available (Verberk, van Noordwijk and Hildrew, 2013). Nevertheless, for several reasons that I now expose, how species traits influence their responses to LUCC remains unclear.
First, there is a lack of comprehensive understanding about which traits are important in shaping species responses to climate change. Wheatley et al. (2017) compared different published climate change vulnerability assessment frameworks, some of which trait-based, some trend-based, and some incorporating elements of both (hybrid). They found that the different frameworks, applied to the same set of species, did not yield consensual outputs and classified species inconsistently into different risk categories. Their work underlines that currently, trend-based vulnerability assessments perform better at identifying species at risks from climate change than trait-based approaches. This study highlights the current lack of unanimous understanding as to which traits to consider, and how, in vulnerability assessments. More broadly, their study stresses the need to clarify our understanding of how response traits to climate change act across different taxa. Wheatley et al. (2017)’s finding that there is no consensus across assessment frameworks might be explained by the fact that frameworks were initially designed and tested for a particular taxon – generally at the class level or lower ranks –, and do not hold when applied to other taxa. They nevertheless argue that frameworks should be universally applicable. Their findings put into question to our current ability to extrapolate the knowledge of response traits gathered for certain taxa to other taxonomic groups. To my knowledge, comparative studies looking at whether response traits to LUCC differ across taxonomic groups (at ranks higher than class), experiencing the same threat levels under similar conditions, are rare. The picture becomes even more complex when different studies find contradicting results within a taxon, such as was the case in some invertebrate species (larger body size having been found to influence species responses to land-use change both negatively (ref) and positively (ref)). The work by Bartomeus et al. (2018) further emphasises the idea that unless similar response traits to a threat are identified consistently across different systems and taxa, our ability to use traits as predictors of vulnerability or extinction risk remains limited. For these reasons, it is necessary to conduct comparative analyses across taxa, to identify response traits, verify whether they are conserved across species and whether they have the same importance in shaping responses across taxonomic groups and geographical areas. 
Second, another difficulty when identifying response traits is that different threats can be acting on the studied ecological community, so that observed modifications stem from the interactions of diverse response traits (Gonzalez-Suarez, Gomez and Revilla, 2013). Response traits must be identified for a single threat while controlling for others, before investigating potential interacting effects. Nevertheless, this is difficult to achieve when using global empirical data. Moreover, the importance of response traits may vary geographically. To conclude, potential taxon-, threat- and geographical dependence of response traits to land-use or climate change makes it difficult to generalise patterns observed at local scales. This stresses the need to conduct global, cross-taxon studies to verify whether empirical evidence supports the generalisation of any response trait.

\section{Land-use and climate change, functional diversity and the disruption of ecosystem services}
Response traits allow to understand and predict how environmental pressures are likely to modify ecological assemblages (changes in species richness and abundance). These alterations can lead to modifications in functional diversity (the diversity and variability of traits in a community). Functional diversity indices are interesting for at least two reasons. First, they can inform on how disturbances affect trait community composition. Second, as functional traits relate to ecosystem functions, measures of functional diversity can relate to ecosystem functioning. I will develop these two points in more detail further down. 

\subsection{Impact of land-use on the functional richness – species richness relationships}
Several indices have been developed in the recent years to estimate diverse components of functional diversity (Schleuter et al., 2010). Functional richness, functional divergence (or dispersion) and functional evenness constitute cornerstone metrics that each quantify different aspects of functional diversity. Functional richness is conceptually similar to species richness but aims at reflecting the number of individual functional units across species in a community rather than quantifying the number of singular species. In other words, functional richness metrics estimate the amount of niche space that species occupy (Carmona et al., 2016). Several metrics have been developed to quantity functional richness alone (Legras, Loiseau and Gaertner, 2018). In this work, I use the dendrogram-based index developed by Petchey and Gaston (2002).
Other indices.
Functional richness indices have been shown to covary with species richness. In experimental studies and natural communities, a positive correlation between these metrics is often found (Petchey and Gaston, 2002). For this reason, examining whether functional richness indices inform on community dynamics differently from species richness is an important question to elucidate. Indeed, if species richness is as informative as functional richness, the latter is not worth measuring: species richness is then a proxy for functional richness. This question was at the heart of the study conducted by Cadotte, Carscadden and Mirotchnick (2011). By reviewing the literature, they found that the relationship between functional richness and species richness is context dependent, and that the shape of the relationship notably depends on the amount of functional redundancy in the community. In communities with a high degree of functional redundancy, functions can be maintained despite species loss. On the other hand, the loss or gain of functionally diverse species can lead to marked variations in functional richness (Figure).
As anthropogenic land-uses globally negatively impact local species richness (Newbold et al., 2015), decreases in functional richness of local ecological communities are likely to take place, particularly in communities with low functional redundancy. Flynn et al. (2009) showed that the functional richness of bird, mammal and plant communities located in the Western hemisphere decreased because of agricultural intensification. In other words, land-use intensification impacted the functional richness-species richness relationship. Mayfield et al. (2010) also showed that the relationship between species richness and functional richness could be affected in different ways by human land-uses. They proposed diverse mechanisms building upon community assembly processes to explain how land-uses may influence species richness – functional richness trajectories. 
The recent development of functional indices, synthesising the diversity of functions in a community, reflects the importance of understanding how anthropogenic pressures will modify ecosystem processes. In the field of biodiversity-ecosystem functioning relationships, it is now well established that higher species diversity is associated with higher ecosystem productivity and stability, better use of limiting resource, as well as better resistance to biological invasions (Tilman at al 2014). I now explore the links between functional diversity indices and ecosystem functioning in more details.

\subsection{Links between functional diversity and ecosystem functioning}
Early experiments investigating the relationships between functional composition and ecosystem functioning classified species in broad functional groups; ecosystem functioning was measured in various ways, depending on the studied system. A higher number of functional groups was correlated to better ecosystem performance in several studies (References). The development of other functional diversity indices allowed for comparisons of diverse predictors of ecosystem functioning. Studies conducted on various systems showed that functional richness performed better at predicting ecosystem functioning than taxonomic richness (Díaz and Cabido, 2001; Flynn et al., 2011; Abonyi, Horváth and Ptacnik, 2018). All these results led to functional diversity emerging as an important predictor of ecosystem functioning. 
The use of functional diversity indices as predictors of ecosystem functioning raises two important points. First, ecosystem functioning should be clearly defined within the study system. Second, the functional traits, from which the processes of interest originate, should be identified and included in the calculation of functional indices. Cadotte, Carscadden and Mirotchnick (2011) underline that point; they highlight that traits should be linked with ecosystem functions for functional diversity indices to be useful. In other words, only relevant effect traits should be considered. Moreover, a larger number of effect traits can lead to higher functional differentiation among species, with potential impacts on the metrics. A careful selection of effect traits is thus vital for functional diversity indices to be informative on ecosystem processes.
If effect traits inform on ecosystem processes, response traits mediate species responses to environmental change. As such, the link between environmental pressures and ecosystem functioning is conceptually realised with both response and effect traits. Specifically, the purpose of the “response-effect” framework is to is to understand how environmental changes alter ecosystem functioning by disentangling traits that render species sensitive to a threat (response traits) from traits that impact functioning (effect traits). A modification in the composition of a community could affect ecosystem functioning in two ways: functions can be lost directly through the removal of species, triggered by response traits (nestedness); and indirectly, functions can be affected by the resulting shift in composition (turnover). The response-effect framework relies on identified response traits to provide a mechanistic understanding of how disturbances modify the trait composition of communities, and how these changes link to alterations in functioning. Changes in functioning are driven by changes in the effect trait composition (effect traits being those that are involved in ecosystem functioning). When response and effect traits are similar, changes in ecosystem functioning are predicted by changes in species composition (direct effects of species loss or gain). In that case, overall functional diversity correlates with ecosystem functioning. However, when response and effect traits are decoupled, changes in functioning are defined by the shifts in effect trait composition only (indirect effects of species loss or gain). 
The application of the response-effect framework to real animal communities has been hindered by several issues (Luck et al 2012; Bartomeus). Luck et al. (2012) proposed a new trait-based framework to link environmental changes to ecosystem services in vertebrate species. They underline the need to develop robust and broadly applicable methods. 
Efforts to link drivers of change and ecosystem function responses have been disparate across taxonomic groups, with a major focus on plants and invertebrates in the past years. Indeed, Hevia et al. (2017) showed in a metanalysis that most studies investigating how species traits mediate the impacts of stressors on ecosystem processes focused on plants and invertebrates, such that there is an existing taxonomic bias in this area. Vegetation and invertebrates both represented an approximate 40\% of the sampled papers, whereas only 17\% were dedicated to vertebrates. Their metanalysis also shed light on other biases, such as the spatial scale of the papers, with most sampled studies being conducted at local or national scales. Therefore, although terrestrial vertebrates have a major cultural, economic and functional importance (Hocking, Babbitt and Hocking, 2014; Whelan, Şekercioğlu and Wenny, 2015; Ratto et al., 2018) and are over-represented in the overall biodiversity literature compared to other taxa (Titley, Snaddon and Turner, 2017), how disturbances affect the services they provide has not been extensively explored compared to other taxa. Efforts to understand ecosystem services provided by terrestrial vertebrates have mainly focused on pest control, seed dispersion, and protein provisioning. To understand how anthropogenic pressures may impact ecosystem processes sustained by vertebrate communities at global scales, there is a need to assess whether LUCC significantly affects the functional diversity of vertebrate communities, and, in particular, the effect trait composition; and to verify whether effect trait composition predicts ecosystem processes, and is, as such, a relevant measure for conservation and mitigation.

To conclude, examining how vertebrate species traits influence their responses to LUCC is the first step to (1) elucidate which traits are likely to put species at greater risk, and find out whether it is possible to generalise patterns across vertebrate species; (2) investigate whether future biodiversity declines triggered by these anthropogenic changes are likely to disrupt important ecosystem functions. The work I have achieved so far focuses on land-use change at global scales and aims at investigating the questions detailed below.

\section{Research questions and hypotheses}
As underlined in the introduction, functional traits can provide a mechanistic understanding of how environmental stressors affect both ecological assemblages and ecosystem processes. As such, they convey information most relevant to conservation policies. According to Hekkala and Roberge (2018), global assessments of how land-use change affects vertebrate functional diversity may have been limited so far by the amount of ecological information required to conduct such analyses, notably by the availability of species traits. There exist published databases of species traits (e.g. Pantheria: Jones et al., 2009; Myhrvold et al., 2015; AmphiBIO: Oliveira et al., 2017) but despite these collation efforts, some taxa remain under sampled  (Newbold, unpublished manuscript). For this project, I will collate information on vertebrate traits prior to conducting any analysis, and I will impute missing values to increase the trait coverage across species. 
Because obtaining longitudinal information on compositional changes can be difficult, the effects of land-use change throughout this project will be studied using a ‘space for time’ substitution, whereby a spatial gradient is used as a proxy for temporal dynamics (De Palma et al., 2018). As such, the following analyses will build upon the PREDICTS database, a large collated dataset of species occurrence and abundance around the world across different land-uses (Hudson et al., 2014, 2017). To date, this database constitutes the most comprehensive global collection of biodiversity samples across different land-uses. It comprises 666 studies, each of which recording the occurrence and/or abundance of species at different sites. Each site is classified into a land-use category; the land-use gradient encompasses primary vegetation, secondary vegetation, plantation forest, cropland, pasture and urban. Primary vegetation refers to native vegetation undisturbed since its development under current climatic conditions. Where primary vegetation was destroyed (either by human actions or natural causes), recovering vegetation forms are referred to as secondary vegetation. Plantation forest, cropland and pasture refer to agricultural areas (crop trees grown for human purposes, biofuels and herbaceous crops, and areas grazed by livestock). Using this database, I aim to address the two following questions.

\subsection{How does land-use affect the functional diversity of vertebrate communities?}

In this part of the project, I aim to investigate how land-use change affects the trait composition of vertebrate communities. I hypothesise that by reducing local habitat heterogeneity, human-dominated land-uses promote functional homogenisation, whereby the similarity in trait composition across assemblages increases through the loss of certain functions (H.1). This hypothesis relies on the idea that strong environmental filtering will disproportionately remove certain functional types. To test this hypothesis, I will use various indices of functional diversity and functional $\beta$-diversity.
First, I will calculate three cornerstone indices of functional diversity. For each PREDICTS site, I will measure functional richness, functional divergence and functional evenness. Specifically:\\
-	I expect functional richness to decrease in more human-dominated land-uses, with habitat filtering reducing the amount of utilised trait space. Because functional richness is often correlated with species richness, I will investigate whether, for a given species richness SR, the functional richness FR is predicted to be lower for sites under higher human land-use intensity (H.1.1). Assuming SR and FR are positively correlated for each land-use type, I expect the slope of the relationship to be lower for sites located within human-dominated land-uses (H.1.1). \\
-	I also expect functional evenness and divergence to decrease along the land-use gradient, with more species within human-dominated land-use communities over-utilising central parts of the functional trait space (H.1.2). I expect to observe a convergence of trait values. 

I will test whether observed effects are consistent across vertebrate classes and contingent on the geographical area. These indices will give insights into how land-use alters overall functional diversity across PREDICTS studies. Nevertheless, such indices are insensitive to changes in functional composition. Indeed, a decrease in functional richness could be explained by either functional turnover, whereby certain trait values are replaced by others, or by functional nestedness, whereby certain functions are lost (Figure 1.). Functional diversity indices will therefore not be informative as to what drives changes in functional composition. 

To further investigate this point, I will use functional $\beta$-diversity measures. $\beta$-diversity indices allow to quantify the dissimilarity in functional composition across sites. Baselga (2010) developed an approach to partition total dissimilarity into one component accounting for functional turnover (potentially stemming from species turnover) and one component accounting for functional nestedness (potential loss or gain of functional trait space relating to the loss or gain of species). I expect nestedness to drive decreases in functional richness (H.1.3) with certain functions being disproportionally lost.


\subsection{Which vertebrate traits confer sensitivity to land-use change?}
The goal of this second analysis is to identify response traits to land-use change in vertebrate species. As opposed to the previous analysis, species traits will be used as explanatory variables. I aim to assess the individual effects of traits or trait combinations on species’ sensitivity to land-use. Specifically, I hypothesize that, with increasing intensity of human land-use:
\begin{itemize}
\item Species with longer generation length are impacted more negatively than shorter-lived species. The rationale behind this hypothesis is that declining population trends in human-dominated land-uses could be compensated more rapidly in species with shorter generation lengths.
\item Larger species (higher body masses) are impacted more negatively than smaller species. Indeed, a general ecological rule states that species with higher body masses have lower local population densities (Santini, Isaac and Ficetola, 2018); I hypothesize that higher body masses compromise species’ persistence in disturbed habitats as those species will tend to be rarer  than species with lower body mass (but see Vermeij and Grosberg (2018)).
\item Species with larger litter or clutch sizes respond less negatively to land-use change than species with smaller litter or clutch sizes. Indeed, declining population trends in human-dominated land-uses could be compensated more in species with larger reproductive outputs.
\item Specialist species, that have stricter requirements either in their habitat preferences or in their diet, are impacted more negatively than generalists. Indeed, such species could be more dependent on particular food sources or habitats that may be impacted negatively by land-use changes.
\item Narrowly distributed species are impacted more negatively than species with larger range sizes. I hypothesize that narrowly distributed species have stricter habitat requirements and lower breadth in the dimensions of their fundamental niche, making them less able to cope with altered habitats than more broadly distributed species.
\end{itemize}